\subsection{Класс HeatEquation}
\begin{verbatim}
package ru.nsu.mmf.g16121.ddt.math;

import java.io.FileNotFoundException;
import java.io.PrintWriter;

import static ru.nsu.mmf.g16121.ddt.main.Main.*;

public class HeatEquations {

    private static final double stepX = (rightBound - leftBound) /
            NUMBERS_COUNT_OF_GRID_BY_X;
    private static final double stepT = (rightBound - leftBound) /
            NUMBERS_COUNT_OF_GRID_BY_T;

    private static final double COURANT_NUMBER_ANALOGUE =
            2.0 * COEFFICIENT_AT_SECOND_DERIVATIVE * stepT /
            Math.pow(stepX, 2);

    /**
     * @param rightPart - this is array of the right part of linear
     *                  equation system Au = rightPart, where A is
     *                  triDiagonal matrix with coeff:
     *                  diag - is coefficient of the matrix A on the 
     *                  diagonal;
     *                  overDiagonal - the coefficient of the matrix
     *                  And the overdiagonal and subdiagonal;
     *                  <p>
     * @return solution matrix of a linear system.
     */

    private static double[] sweepMethodWithConstCoef(
    double[] rightPart) {
        double diag = 1.0 + COURANT_NUMBER_ANALOGUE;
        double overDiagonal = -COURANT_NUMBER_ANALOGUE * 0.5;

        double[] u = new double[rightPart.length];

        double[] alpha = new double[rightPart.length - 1];
        double[] beta = new double[rightPart.length - 1];

        alpha[rightPart.length - 2] = -overDiagonal / diag;
        beta[rightPart.length - 2] = rightPart[rightPart.length - 1] /
        diag;

        for (int i = rightPart.length - 3; i >= 0; i--) {
            alpha[i] = -(overDiagonal / (diag + overDiagonal * 
            alpha[i + 1]));
            beta[i] = ((rightPart[i + 1] - beta[i + 1] * overDiagonal)
            /(diag + overDiagonal * alpha[i + 1]));
        }

        u[0] = ((rightPart[0] - overDiagonal * beta[0]) / (diag +
                overDiagonal * alpha[0]));
        for (int i = 1; i < rightPart.length; i++) {
            u[i] = alpha[i - 1] * u[i - 1] + beta[i - 1];
        }
        return u;
    }

    /**
     * Tn this method (<>writeInTxt</>) we write points surface in 
     * txt file: x - in 1st column, t - in 2d column,
     * exact value of the func - 3d column and our func value in 
     * 4d column (for gnuplot)
     */

    private static void writeForGnuplot(double[][] u) throws
    FileNotFoundException {
        PrintWriter writer = new
        PrintWriter("functions_for_Gnuplot.txt");

        double x;
        double t = leftBound;
        for (int i = 0; i <= NUMBERS_COUNT_OF_GRID_BY_T; i++) {
            x = leftBound;
            for (int j = 0; j <= NUMBERS_COUNT_OF_GRID_BY_X; j++) {
                writer.println(x + "\t" + t + "\t" + u(x, t) + "\t" 
                + u[i][j]);
                x += stepX;
            }
            t += stepT;
        }
        writer.close();
    }

    private static void writeResultForPython(double[][] u) throws
    FileNotFoundException {
        PrintWriter writer = new PrintWriter("result.txt");

        writer.print("[[" + (int) leftBound + ", " + (int) rightBound
                + ", " + (NUMBERS_COUNT_OF_GRID_BY_X + 1) + "],");
        writer.print("[" + (int) leftBound + ", " + (int) rightBound
                + ", " + (NUMBERS_COUNT_OF_GRID_BY_T + 1) + "],");

        double x;
        double t = leftBound;

        writer.print("[");
        for (int i = 0; i <= NUMBERS_COUNT_OF_GRID_BY_T; i++) {
            x = leftBound;
            writer.print("[");
            for (int j = 0; j < NUMBERS_COUNT_OF_GRID_BY_X; j++) {
                writer.print(u[i][j] + ",");
                x += stepX;
            }
            writer.print(u[i][NUMBERS_COUNT_OF_GRID_BY_X]);
            if (i == NUMBERS_COUNT_OF_GRID_BY_T) {
                writer.print("]");
            } else {
                writer.print("],");
            }
            t += stepT;
        }
        writer.print("]]");
        writer.close();
    }

    private static void writeMainFuncForPython() throws
    FileNotFoundException {
        PrintWriter writer = new PrintWriter("mainFunc.txt");

        writer.print("[[" + (int) leftBound + ", " + (int) rightBound
                + ", " + (NUMBERS_COUNT_OF_GRID_BY_X + 1) + "],");
        writer.print("[" + (int) leftBound + ", " + (int) rightBound
                + ", " + (NUMBERS_COUNT_OF_GRID_BY_T + 1) + "],");

        double x;
        double t = leftBound;

        writer.print("[");
        for (int i = 0; i <= NUMBERS_COUNT_OF_GRID_BY_T; i++) {
            x = leftBound;
            writer.print("[");
            for (int j = 0; j < NUMBERS_COUNT_OF_GRID_BY_X; j++) {
                writer.print(u(x, t) + ",");
                x += stepX;
            }
            writer.print(u(x, t));
            if (i == NUMBERS_COUNT_OF_GRID_BY_T) {
                writer.print("]");
            } else {
                writer.print("],");
            }
            t += stepT;
        }
        writer.print("]]");
        writer.close();
    }

    private static double maxError(double[][] u) {

        double x = leftBound;
        double t = leftBound;
        double max = Math.abs(u[0][0] - u(x, t));
        for (int i = 0; i <= NUMBERS_COUNT_OF_GRID_BY_T; i++) {
            x = leftBound;
            for (int j = 0; j <= NUMBERS_COUNT_OF_GRID_BY_X; j++) {
                double error = Math.abs(u(x, t) - u[i][j]);
                if (error > max) {
                    max = error;
                }
                x += stepX;
            }
            t += stepT;
        }
        return max;
    }

    /**
     * Tn this method (<>solveHeatEquation</>) solve the heat
     * equation solves the heat equation using the Euler method,
     * with initial conditions, and writes data to a text file 
     * to display the result.
     */

    public static void solveHeatEquation() 
    throws FileNotFoundException {
        double[][] u = new double[NUMBERS_COUNT_OF_GRID_BY_T + 1]
                [NUMBERS_COUNT_OF_GRID_BY_X + 1];

        //The first row of the matrix is filled by the initial data.
        double x = leftBound;
        for (int i = 0; i <= NUMBERS_COUNT_OF_GRID_BY_X; i++) {
            u[0][i] = mu(x);
            x += stepX;
        }

        //The first and second columns are filled with source data.
        double t = leftBound;
        for (int j = 0; j <= NUMBERS_COUNT_OF_GRID_BY_T; j++) {
            u[j][0] = mu1(t);
            u[j][NUMBERS_COUNT_OF_GRID_BY_X] = mu2(t);
            t += stepT;
        }

        //build the right part for the sweep method
        t = leftBound;
        for (int j = 0; j <= NUMBERS_COUNT_OF_GRID_BY_T - 1; j++) {
            x = leftBound + stepX;
            double[] rightPart = 
            new double[NUMBERS_COUNT_OF_GRID_BY_X - 1];
            for (int i = 0; i <= NUMBERS_COUNT_OF_GRID_BY_X - 2; i++) {
                rightPart[i] = u[j][i + 1] + stepT * f(x, t + stepT);
                x += stepX;
            }
            rightPart[0] += COURANT_NUMBER_ANALOGUE * 0.5 * 
            mu1(t + stepT);
            rightPart[NUMBERS_COUNT_OF_GRID_BY_X - 2] +=
            COURANT_NUMBER_ANALOGUE * 0.5 * mu2(t + stepT);

            //fill the rest of the matrix
            System.arraycopy(sweepMethodWithConstCoef(rightPart),
                    0, u[j + 1], 1, NUMBERS_COUNT_OF_GRID_BY_X - 1);
            t += stepT;
        }

        //write in the txt for display the result
        writeForGnuplot(u);
        writeMainFuncForPython();
        writeResultForPython(u);
        System.out.println("Max error = " + maxError(u));
    }
}
\end{verbatim}
\subsection{Класс  Main}
\begin{verbatim}
package ru.nsu.mmf.g16121.ddt.main;

import ru.nsu.mmf.g16121.ddt.math.HeatEquations;

import java.io.IOException;

public class Main {
    public static final double leftBound = 0;
    public static final double rightBound = 1;

    public static final int NUMBERS_COUNT_OF_GRID_BY_X = 10;
    public static final int NUMBERS_COUNT_OF_GRID_BY_T = 20;

    public static final double 
            COEFFICIENT_AT_SECOND_DERIVATIVE = 0.021;

    public static double f(double x, double t) {
        return (x + 2.0 * t - Math.exp(x) + 
                COEFFICIENT_AT_SECOND_DERIVATIVE * (12.0 * 
                        Math.pow(x, 2) - 4.0 + t * Math.exp(x)));
    }

    public static double mu(double x) {
        return -Math.pow(x, 4) + 2.0 * Math.pow(x, 2);
    }

    public static double mu1(double t) {
        
        return Math.pow(t, 2) - t;
    }

    public static double mu2(double t) {
        return 1 + t + Math.pow(t, 2) - t * Math.E;
    }

    public static double u(double x, double t) {
        return -Math.pow(x, 4) + 2.0 * Math.pow(x, 2) 
                + t * x + Math.pow(t, 2) - t * Math.exp(x);
    }

    public static void main(String[] args) throws IOException {
        HeatEquations.solveHeatEquation();

        Runtime.getRuntime().exec("python3 vizualization.py");
        Runtime.getRuntime().exec("python3 vizualization2.py");
    }
}

\end{verbatim}